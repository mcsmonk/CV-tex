%------------------------
% Resume Template
% Author : Sunghyun Jin
% License : MIT
%------------------------

\documentclass[a4paper,20pt]{article}

\usepackage{kotex}

\usepackage{latexsym}
\usepackage[empty]{fullpage}
\usepackage{titlesec}
\usepackage{marvosym}
\usepackage[usenames,dvipsnames]{color}
\usepackage{verbatim}
\usepackage{enumitem}
\usepackage[pdftex]{hyperref}
\usepackage{fancyhdr}

\pagestyle{fancy}
\fancyhf{} % clear all header and footer fields
\fancyfoot{}
\renewcommand{\headrulewidth}{0pt}
\renewcommand{\footrulewidth}{0pt}

% Adjust margins
\addtolength{\oddsidemargin}{-0.530in}
\addtolength{\evensidemargin}{-0.375in}
\addtolength{\textwidth}{1in}
\addtolength{\topmargin}{-.45in}
\addtolength{\textheight}{1in}

\urlstyle{rm}

\raggedbottom
\raggedright
\setlength{\tabcolsep}{0in}

% Sections formatting
\titleformat{\section}{
  \vspace{-10pt}\scshape\raggedright\large
}{}{0em}{}[\color{black}\titlerule \vspace{-6pt}]


%-----------------------------
%%%%%%  CV STARTS HERE  %%%%%%

\begin{document}

\begin{tabular*}{\textwidth}{l@{\extracolsep{\fill}}r}
  \textbf{{\LARGE Sunghyun Jin}}\vspace{8pt}\\
  Security Software Engineer & \href{mailto:mcsmonk@gmail.com}{mcsmonk@gmail.com}\\
  Software Development Team & ORCID(\href{https://orcid.org/0000-0002-9521-0937}{0000-0002-9521-0937})\\
  Memory Business, Device Solutions & \href{https://sunghyunjin.com}{https://sunghyunjin.com}\\
  Samsung Electronics & last modified : 19 Sep 2024\\
\end{tabular*}


\iffalse
\section{\textbf{Research Interests}}
\begin{itemize}
    \item {Cryptographic Engineering}
    \vspace{-4pt}
    \begin{itemize}
        \item {Side-Channel Analysis and its Countermeasures}
        \vspace{-2pt}
        \item {Cryptosystem Implementation Optimization}
    \end{itemize}
\end{itemize}
\fi


\section{\textbf{Experience}}
\begin{itemize}
    \item {\textbf{Staff engineer}, Software Development Team, Memory Business, Device Solutions, Samsung Electronics, Hwaseong-si, Gyeonggi-do, Republic of Korea, Sep 2022 - \textbf{Present}}
        \vspace{-6pt}
        \begin{itemize}
            \item {Stograge Security, Attestation, Endorsement, Root-of-Trust, PCI Security, PKI}
            \item {DMTF SPDM, TCG DICE, PCI-SIG CMA/IDE/TDISP}
        \end{itemize}
    \item {\textbf{Graduate student researcher}, Center for Information Security Technologies (CIST), Institute of Cyber Security and Privacy (ICSP), Korea University, Seoul, Republic of Korea, Mar 2015 - Aug 2022}
        \vspace{-6pt}
        \begin{itemize}
            \item {Cryptographic Engineering, Side-Channel Analysis and its Countermeasures, Implementation Optimization, Machine/Deep Learning-based Cryptanalysis}
        \end{itemize}
    % \textbf{Present}
    %\vspace{-6pt}
    %\item {Graduate student researcher, Center for Information Security Technologies (CIST), Institute of Cyber Security and Privacy (ICSP), Korea University, Seoul, Republic of Korea \hfill Mar 2017 - \textbf{Present}}
    %\vspace{-6pt}
    %\item {(Mar 2017 - \textbf{Present})\hfill Graduate student researcher, Center for Information Security Technologies (CIST), Institute of Cyber Security and Privacy (ICSP), Korea University, Seoul, Republic of Korea}
\end{itemize}

\iffalse
        \vspace{-6pt}
        \begin{description}
            \item {Stograge Security, Attestation, Root-of-Trust, PCI Security}
        \end{description}
\fi

\section{\textbf{Education}}
\begin{itemize}
    \item {\textbf{Korea University} \hfill Seoul, Republic of Korea}
        \vspace{-6pt}
        \begin{description}
            \item {\textit{Ph.D. in Information Security} \hfill Mar 2017 - Aug 2022}
            \vspace{-2pt}
            \item {Thesis: Algebraic Side-Channel Attack against ECDSA Employing Table-based Scalar Multiplication}
            \vspace{-2pt}
            \item {Supervised by Seokhie Hong and HeeSeok Kim}
            \vspace{-2pt}
            \item {Research on side-channel analysis and its countermeasures}
                %\begin{itemize}
                %\end{itemize}
        \end{description}
    \item {\textbf{Korea University} \hfill Seoul, Republic of Korea}
        \vspace{-6pt}
        \begin{description}
            \item {\textit{M.E. in Information Security} \hfill Mar 2015 - Feb 2017}
            \vspace{-2pt}
            \item {Thesis: 프로파일링 단계에서 파형 선별을 통한 템플릿 공격의 성능 향상}
            \vspace{-2pt}
            \item {Supervised by Seokhie Hong}
            \vspace{-2pt}
            \item {Research on side-channel analysis and its countermeasures}
        \end{description}
    \item {\textbf{University of Seoul} \hfill Seoul, Republic of Korea}
        \vspace{-6pt}
        \begin{description}
            \item {\textit{B.S. in Mathematics and Computer Science} \hfill Mar 2009 - Feb 2015}
            \vspace{-2pt}
            \item {mandatory serve in the army from Jan 2010 to Nov 2011}
            %{\scriptsize \textit{ \footnotesize{\newline{}\textbf{Courses:} Operating Systems, Data Structures, Analysis Of Algorithms, Artificial Intelligence, Machine Learning, Networking, Databases}}}
        \end{description}
\end{itemize}



\section{\textbf{Publication}}
\begin{itemize}
\iffalse
    \item {Work in Progress}
        \vspace{-6pt}
        \begin{enumerate}
            \item {SangYun Jung, \textbf{Sunghyun Jin}, HeeSeok Kim. Title, under review}
            \vspace{-2pt}
            % (A tentative title)
            % submitted to {}
            % under review
        \end{enumerate}
\fi
        
    \item {International Journal}
        \vspace{-6pt}
        \begin{enumerate}
            \item {SangYun Jung, \textbf{Sunghyun Jin}, HeeSeok Kim. A Novel Side-Channel Archive Framework Using Deep Learning-based Leakage Compression. IEEE Access, 2024}
            \item {\textbf{Sunghyun Jin}, Sung Min Cho, HeeSeok Kim, Seokhie Hong. Enhanced Side-Channel Analysis on ECDSA Employing Fixed-Base Comb Method, IEEE Transactions on Computers 71.9: 2341-2350, 2022}
            \item {\textbf{Sunghyun Jin}, Philip Johansson, HeeSeok Kim, Seokhie Hong. Enhancing Time-Frequency Analysis with Zero-mean Preprocessing. Sensors 22.7, 2022}
            \vspace{-2pt}
            \item {\textbf{Sunghyun Jin}, Sangyub Lee, Sung Min Cho, HeeSeok Kim, Seokhie Hong. Novel Key Recovery Attack on Secure ECDSA Implementation by Exploiting Collisions between Unknown Entries. IACR Transactions on Cryptographic Hardware and Embedded Systems 2021(4): 1-26, 2021}
            \vspace{-2pt}
            \begin{itemize}
                \item {Presented in the annual Conference on Cryptographic Hardware and Embedded Systems, CHES 2021}
            \end{itemize}
            \vspace{-2pt}
            \item {\textbf{Sunghyun Jin}, Suhri Kim, HeeSeok Kim, Seokhie Hong. Recent advances in deep learning-based sde-channel analysis. ETRI Journal 42.2: 292-304, 2020}
            \vspace{-2pt}
            \item {Soojung An, Suhri Kim, \textbf{Sunghyun Jin}, HanBit Kim, HeeSeok Kim. Single Trace Side Channel Analysis on NTRU Implementation. Applied Sciences 8.11, 2018}
            \vspace{-2pt}
            \item {Sung Min Cho, \textbf{Sunghyun Jin}, HeeSeok Kim. Side-Channel Vulnerabilities of Unified Point Addition on Binary Huff Curve and Its Countermeasure. Applied Sciences 8.10, 2018}
            \vspace{-2pt}
        \end{enumerate}
        
    \item {International Conference}
        \vspace{-6pt}
        \begin{enumerate}
            \item {Suhri kim, \textbf{Sunghyun Jin}, Yechan Lee, Byeonggyu Park, Hanbit Kim, Seokhie Hong. Single Trace Side Channel Analysis on Quantum Key Distribution. International Conference on Information and Communication Technology Convergence (ICTC), 2018}
            \vspace{-2pt}
        \end{enumerate}
        
    \item {Domestic Journal (Republic of Korea)}
        \vspace{-6pt}
        \begin{enumerate}
            \item {SangYun Jung, \textbf{Sunghyun Jin}, HeeSeok Kim. Side-Channel Archive Framework Using Deep Learning-Based Leakage Compression. Journal of the Korea Institute of Information Security and Cryptology, Vol. 34, No. 3, 2024}
            \item {Yechan Lee, \textbf{Sunghyun Jin}, Hanbit Kim, HeeSeok Kim, Seokhie Hong. New Higher-Order Differential Computation Analysis on Masked White-Box AES. Journal of the Korea Institute of Information Security and Cryptology, Vol. 30, No. 1, 2020}
            \vspace{-2pt}
            \item {Donggeun Kwon, \textbf{Sunghyun Jin}, HeeSeok Kim, Seokhie Hong. Improving Non-Profiled Side-Channel Analysis Using Auto-Encoder Based Noise Reduction Preprocessing. Journal of the Korea Institute of Information Security and Cryptology, Vol. 29, No. 3, 2019}
            \vspace{-2pt}
            \item {Byeonggyu Park, Suhri kim, Hanbit Kim, \textbf{Sunghyun Jin}, HeeSeok Kim, Seokhie Hong. Single Trace Analysis against HyMES by Exploitation of Joint Distributions of Leakages. Journal of the Korea Institute of Information Security and Cryptology, Vol. 28, No. 5, 2018}
            \vspace{-2pt}
            \item {Soojung An, Suhri kim, \textbf{Sunghyun Jin}, Hanbit Kim, HeeSeok Kim, Seokhie Hong. Single Trace Side Channel Analysis on NTRUEncrypt Implementation. Journal of the Korea Institute of Information Security and Cryptology, Vol. 28, No. 5, 2018}
            \vspace{-2pt}
            \item {Gayeong Ko, \textbf{Sunghyun Jin}, Hanbit Kim, HeeSeok Kim, Seokhie Hong. Improved Side Channel Analysis Using Power Consumption Table. Journal of the Korea Institute of Information Security and Cryptology, Vol. 27. No. 4, 2017}
            \vspace{-2pt}
            \item {\textbf{Sunghyun Jin}, Taewon Kim, HeeSeok Kim, Seokhie Hong. Power Trace Selection Method in Template Profiling Phase for Improvements of Template Attack. Journal of the Korea Institute of Information Security and Cryptology, Vol. 27, No. 1, 2017}
            \vspace{-2pt}
        \end{enumerate}
        
    \item {Domestic Conference (Republic of Korea)}
        \vspace{-6pt}
        \begin{enumerate}
            \item {Daehyeon Bae, \textbf{Sunghyun Jin}, HeeSeok Kim, Seokhie Hong. 병렬 회로의 잡음 분석을 통한 부채널 분석 및 활용 방안. Conference on Information Security and Cryptography Summer (CISC-S), Korea Institute of Information Security and Cryptology, 2022.06.17.}
            \vspace{-2pt}
            \item {\textbf{Sunghyun Jin}, HeeSeok Kim, Seokhie Hong. 다중 작업 학습을 이용한 딥러닝 기반 부채널 분석 연구. Conference on Information Security and Cryptography Summer (CISC-S), Korea Institute of Information Security and Cryptology, 2020.07.15.}
            \vspace{-2pt}
            \item {Donggeun Kwon, \textbf{Sunghyun Jin}, Hanbit Kim, HeeSeok Kim, Seokhie Hong. 차분 딥러닝 분석 성능 향상을 위한 새로운 구조 제안. Conference on Information Security and Cryptography Winter (CISC-W), Korea Institute of Information Security and Cryptology, 2019.11.30.}
            \vspace{-2pt}
            \item {Yechan Lee, \textbf{Sunghyun Jin}, Hanbit Kim, HeeSeok Kim, Seokhie Hong. 동적 분석 기반 공개키 암호알고리즘의 부채널 취약 구현 평가 연구. Conference on Information Security and Cryptography Winter (CISC-W), Korea Institute of Information Security and Cryptology, 2019.11.30.}
            \vspace{-2pt}
            \item {\textbf{Sunghyun Jin}, Donggeun Kwon, HeeSeok Kim, Seokhie Hong. 논프로파일링 환경에서 CPA 성능향상을 위한 딥러닝 기반 Correlation Optimization 기술 연구. Conference on Information Security and Cryptography Summer (CISC-S), Korea Institute of Information Security and Cryptology, 2019.06.20.}
            \vspace{-2pt}
            \item {Donggeun Kwon, \textbf{Sunghyun Jin}, HeeSeok Kim, Seokhie Hong. Generative Adversarial Networks를 이용한 부채널 분석 성능 향상에 관한 연구. Conference on Information Security and Cryptography Summer (CISC-S), Korea Institute of Information Security and Cryptology, 2019.06.20.}
            \vspace{-2pt}
            \item {Yechan Lee, \textbf{Sunghyun Jin}, Hanbit Kim, HeeSeok Kim, Seokhie Hong. 마스킹 화이트 박스 암호에 대한 새로운 2차 차분 계산 분석 기술 연구. Conference on Information Security and Cryptography Summer (CISC-S), Korea Institute of Information Security and Cryptology, 2019.06.20.}
            \vspace{-2pt}
            \item {Donggeun Kwon, \textbf{Sunghyun Jin}, HeeSeok Kim, Seokhie Hong. 오토인코더 기반 딥러닝을 활용한 부채널 분석 노이즈 제거 기술 연구. Conference on Information Security and Cryptography Winter (CISC-W), Korea Institute of Information Security and Cryptology, 2018.12.08.}
            \vspace{-2pt}
        \end{enumerate}
        
    \item {ETC}
        \vspace{-6pt}
        \begin{enumerate}
            \item {\textbf{Sunghyun Jin}, HeeSeok Kim. 딥러닝을 이용한 부채널 분석 기술 연구 동향. Korea Institute of Information Security and Cryptology, Review of Korea Institute of Information Security and Cryptology, page 43-53, Vol. 30(1), Feb 2020 (Korean)}
            \vspace{-2pt}
        \end{enumerate}
\end{itemize}



\section{\textbf{Patent}}
\begin{itemize}
    \item {Seokhie Hong, Soojung An, \textbf{Sunghyun Jin}, DongWon Park. Method for Restorating Prime Number using Thread Information, Device and Computer Readable Medium for Performing the Method. KR1021995070000, 2021-01-06}
    %\vspace{-4pt}
\end{itemize}



\iffalse
\section{\textbf{Project}}
\item {Work in Progress}
    \vspace{-6pt}
    \begin{enumerate}
        \item {blah}
        \vspace{-2pt}
    \end{enumerate}
\begin{itemize}
    \item {blah}
    \vspace{-4pt}
\end{itemize}
\fi



\iffalse
\section{\textbf{Talk}}
    \item {Work in Progress}
        \vspace{-6pt}
        \begin{enumerate}
            \item {blah}
            \vspace{-2pt}
        \end{enumerate}
    \begin{itemize}
        \item {blah}
        \vspace{-4pt}
    \end{itemize}
\fi



\iffalse
\section{\textbf{Teaching}}
\item {Work in Progress}
        \vspace{-6pt}
        \begin{enumerate}
            \item {blah}
            \vspace{-2pt}
        \end{enumerate}
    \begin{itemize}
        \item {blah}
        \vspace{-4pt}
    \end{itemize}
\fi



\section{\textbf{Community Service}}
\begin{itemize}
    \item {Peer Reviewing}
        \vspace{-6pt}
        \begin{itemize}
            \item {Reviewer}
                \vspace{-4pt}
                \begin{enumerate}
                    \item {Security and Communication Networks, Hindawi : 2019}
                \end{enumerate}
            \item {Subreviewer : IACR ASIACRYPT(2019), IACR TCHES(2021-1), MDPI Sensors(2021), IEEE Access(2022), IEEE TCSII(2022), KIISC ICISC(2023)}
                %\vspace{-4pt}
        \end{itemize}
\end{itemize}



\section{\textbf{Honor and Award}}
\begin{itemize}
    \item {Excellence Award in the Theory Category of the National Cryptography Contest, Korea Cryptography Forum, 2021}
    \vspace{-4pt}
    \item {President's Award, Korea University, 2017}
    \vspace{-4pt}
    \item {Grand Prize in the National Cryptography Technology Professional Training Course, National Security Research institute, 2016}
\end{itemize}



\section{\textbf{Reference}}
\begin{itemize}
    \item {Seokhie Hong
           (\href{mailto:shhong@korea.ac.kr}{shhong@korea.ac.kr},
           orcid: \href{https://orcid.org/0000-0001-7506-4023}{0000-0001-7506-4023})
           : Superevisor}
    \vspace{-4pt}
    \item {HeeSeok Kim
           (\href{mailto:80khs@korea.ac.kr}{80khs@korea.ac.kr},
           orcid: \href{https://orcid.org/0000-0001-8137-4810}{0000-0001-8137-4810})
           : Co-supervisor}
\end{itemize}











\iffalse

\newcommand{\resumeItem}[2]{
  \item\small{
    \textbf{#1}{: #2 \vspace{-2pt}}
  }
}

\newcommand{\resumeItemWithoutTitle}[1]{
  \item\small{
    {\vspace{-2pt}}
  }
}

\newcommand{\resumeSubheading}[4]{
  \vspace{-1pt}\item
    \begin{tabular*}{0.97\textwidth}{l@{\extracolsep{\fill}}r}
      \textbf{#1} & #2 \\
      \textit{#3} & \textit{#4} \\
    \end{tabular*}\vspace{-5pt}
}

\newcommand{\resumeSubItem}[2]{\resumeItem{#1}{#2}\vspace{-3pt}}
\renewcommand{\labelitemii}{$\circ$}

\newcommand{\resumeSubHeadingListStart}{\begin{itemize}[leftmargin=*]}
\newcommand{\resumeSubHeadingListEnd}{\end{itemize}}
\newcommand{\resumeItemListStart}{\begin{itemize}}
\newcommand{\resumeItemListEnd}{\end{itemize}\vspace{-5pt}}


\newpage
\section{\textbf{Skills Summary}}
	\resumeSubHeadingListStart
	\resumeSubItem{Languages}{~~~~~~Python, PHP, C++, JavaScript, SQL, Bash, JAVA}
	\resumeSubItem{Frameworks}{~~~~Scikit, NLTK, SpaCy, TensorFlow, Keras, Django, Flask, NodeJS, LAMP}
	\resumeSubItem{Tools}{~~~~~~~~~~~~~~Kubernetes, Docker, GIT, PostgreSQL, MySQL, SQLite}
	\resumeSubItem{Platforms}{~~~~~~~Linux, Web, Windows, Arduino, Raspberry, AWS, GCP, Alibaba Cloud, IBM Cloud}
	\resumeSubItem{Soft Skills}{~~~~~~~Leadership, Event Management, Writing, Public Speaking, Time Management}
\resumeSubHeadingListEnd

\section{Experience}
  \resumeSubHeadingListStart
    \resumeSubheading{Google Summer of Code - Submitty}{Remote}
    {Student Developer (Full-time)}{May 2019 - Sep 2019}
    \resumeItemListStart
        \resumeItem{Discussion Forum Upgrades}
          {Refactor forum for performance to handle large databases.}
          \resumeItem{REST API for Discussion Forum}
          {Symphony \& Twig based Forum parts converted to API-first interface.}
          \resumeItem{Ratchet PHP WebSocket}{Implemented a WebSocket for low-latency real time exchange of posts and thread updates.}
      \resumeItemListEnd
\vspace{-5pt}
    \resumeSubheading
		{DataCamp Inc.}{Remote}
		{Instructor (Part-time, Contractual)}{Dec 2018 -  Present}
		\resumeItemListStart
        \resumeItem{Project Course - Find Movie Similarity from Plot Summaries}
          {Created project based course using Unsupervised learning and natural language processing.}
        \resumeItem{Tutorial - Introduction to Reinforcement Learning}
          {Created tutorial for Q-learning RL algorithm and  concepts.}
        \resumeItem{Impact}{Course has been taken by 250+ students so far with 4.65 average rating.}
		\resumeItemListEnd
\resumeSubHeadingListEnd

\section{Projects}
\resumeSubHeadingListStart
\resumeSubItem{Vison - multimedia search engine (NLP, Search Engine, Web Crawlers, Multimedia Processing)}{(Work in progress) Research oriented, open source, search engine for bringing reverse multimedia search to small \& mid scale enterprises. Tech: Python, NodeJS, Intel OpenVino Toolkit, Selenium, TensorFlow (October '18)}
\vspace{2pt}
\resumeSubItem{Reinforcement Learning based Traffic Control System (Reinforcement Learning, Computer Vision)}{AI model to resolve city traffic around 50\%
faster. Tech: Python, Alibaba Cloud, Raspberry Pi, Arduino, SUMO \& OpenCV. (August '18)}
\vspace{2pt}
\resumeSubItem{Panorama from Satellite Imagery using Distributed Computing (Distributed Computing, Image Processing)}{Images clicked using drones, provided by ISRO were stitched together using distributed public compute nodes, effectively bringing down processing time exponentially. Tech: PHP, C++, Java, Python (March '18)}
\vspace{2pt}
\resumeSubItem{Drag-n-drop machine learning learning environment (Web Development, Machine Learning)}{Scratch like tool for implementing machine learning pipelines along with built in tutorial for each concept. Tech: Python, JavaScript (September '18)}
\vspace{2pt}
\resumeSubItem{Search Engine and Social Network(Web Development, Web Crawler, Search)}{Created from scratch a social network and a search engine based on the idea of integrating Facebook and Google. The launched website was among top 1000 websites in India during 2012-2013. Tech: PHP, MySQL, HTML, CSS, WebSockets, JavaScript, RSS, XML ( May '12)}
\resumeSubHeadingListEnd
\vspace{-5pt}

\section{Volunteer Experience}
  \resumeSubHeadingListStart
	\resumeSubheading
    {Community Lead at Developer Student Clubs NSEC}{Kolkata, India}
    {Conducted online and offline technical \& soft-skills training impacting over 3000 students.}{Jan 2019 - Present}
\vspace{5pt}
    % \vspace{10pt}\textbf{\large{Community Experience}}
    \resumeSubheading
    {Event Organizer at Google Developers Group Kolkata}{Kolkata, India}
    {Organized events, conducted workshops and delivered workshops reaching over 7000 developers.}{Jan 2018 - Present}

\resumeSubHeadingListEnd

\fi

\end{document}